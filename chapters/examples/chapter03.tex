\chapter{Definitionen}
\label{ch:chapter02}
Dieses Kapitel erklärt und beschreibt einige der zentralen Begriffe zum Thema des Hostes und der Berechtigungsstrukturen.
Dadurch soll der Leser ein Grundverständnis erhalten, um die folgenden Kapitel zu verstehen.

%
% Section: Der erste Abschnitt
%

\section{Host/Mainframe}
\label{sec:Host}
Der Host oder als auch Mainframe bekannt ist ein Komplex aus verschiedenen Hochleistungscomputern.
Der Anbieter "`IBM"' definiert diesen dabei wie folgt: 
\newline
\newline
\textit{"`At their core, mainframes are high-performance computers with large amounts of memory and processors that process billions of simple calculations and transactions in real time."'} \cite{Mainframe}
\newline
\newline
Oder übersetzt:
\newline
\newline
\textit{"`Im Kern besteht der Mainframe aus Hochleistungsrechnern, welche über einen großen Speicher verfügen, und in der Lage sind Milliarden von einfachen Prozessen und Transaktion in Echtzeit durchzuführen."'} \cite{Mainframe}
\newline
\newline
Eine ähnliche Definition verwenden das Wirtschaftslexikon Gabler \cite{Main} sowie Cambrige Dictionary \cite{CambMain}
Dabei spielt der Mainframe eine wichtige Rolle in der Finanzindustrie, welche über widerstandsfähigen, sicheren und agilen Servern benötigen.
Dies ist der Fall, weil die Finanzindustrie über viele sensible Daten verfügt.
Daher müssen die Server sicher und widerstandfähig sein, damit diese Daten nicht verloren gehen oder gestohlen werden.
Zudem müssen die Server agil sein, da die Technology und die Regulierungen um den Mainframe sich steht ändern und daher dieser auf dem neuesten Stand sein muss.
Dieser muss nämlich die Regularien vom \ac{VAIT} erfüllen die von der \ac{BaFin} aufgestellt werden.
Diese soll eine konsistente IT-Strategie vorgeben, an welche sich die Unternehmen halten müssen. \cite{Vait}

\section{Berechtigung}
\label{sec:Berechtigung}
Dabei definiert die \ac{NIST}, welche eine Institution von amerikanischer Regierung ist, Berechtigungen wie folgt:
\newline
\newline
\textit{"`The right or a permission that is granted to a system entity to access a system resource."'} \cite{Auth}
\newline
\newline
Dies bedeutet:
\newline
\newline
\textit{"`Das Recht oder die Erlaubnis haben, um auf System Ressourcen einer Systemeinheit zu zugreifen."'} \cite{Auth}
\newline
\newline
Oxford Dictionary gibt eine allgemeinere, aber dennoch ähnlich Definition \cite{AuthOx}.
Dabei bietet The Economic Times eine fast identische Definition wie der von \ac{NIST} \cite{AuthEco}
Im Kontext des Mainframebereiches betrifft dies hauptsächlich das Betrachten und Zugreifen von Dialogmasken bei der Helvetia.
\begin{figure}[h!]
 \centering
 \includegraphics[width=1\textwidth]{gfx/Picture/Dialog.PNG}
 \caption{Beispiel Dialogmaske}
 \label{fig:Dial}
\end{figure}
Die Graphik (\ref{fig:Dial}) zeigt ein solches Dialogmaske.
Anhand dieser Dialogmaske kann man erkennen, dass der Nutzer L895 für die aufgezählten weiteren Dialogmasken (EL 00, EX 00, ...) zumindest die Leseberechtigung hat.
Zudem hat dieser die Schreibberechtigungen auf die Dialogmaske DM 00, da dieser sich in dieser Maske aufhält. 
\newline
\newline
\begin{figure}[h!]
 \centering
 \includegraphics[width=1\textwidth]{gfx/Picture/Berechtigung.PNG}
 \caption{Berechtigungsdialogmaske}
 \label{fig:Berch}
\end{figure}
In dieser Graphik (\ref{fig:Berch}) kann man die Profile und individuellen Berechtigungen sehen, die der Nutzer L895 besitzt.
Dabei sind Profile eine Ansammlung von Berechtigungen.

\section{Berechtigungsstruktur}
\label{sec:Berechtigungsstruktur}
Berechtigungsstruktur besteht aus den Berechtigungen, die im Unterkapitel (\ref{sec:Berechtigung}) definiert werden, und aus der Struktur.
Dabei definiert Oxford Struktur wie folgt:
\newline
\newline
\textit{"`the way in which the parts of something are connected together, arranged or organized; a particular arrangement of parts"'} \cite{Struct}
\newline
\newline
Dies bedeutet so viel wie, dass die Dinge von etwas miteinander verknüpft, angeordnet oder organisiert sind. \cite{Struct}
Cambrige Dictionary bietet eine ähnliche Definition an. \cite {CambStruc} 
\newline
In diesem Zusammenhang bedeutet Berechtigungsstruktur die Verknüpfung, Anordnung oder Organisation von Berechtigungen.
\begin{figure}[h!]
 \centering
 \includegraphics[width=1.25\textwidth]{gfx/Picture/Struktur.PNG}
 \caption{Teilausschnitt der Berechtigungsstruktur der Helvetia}
 \label{fig:Teil}
\end{figure}
Wie man in der Graphik (\ref{fig:Teil}) erkennen kann sind die Profile hierarchisch aufgebaut.
Diese Profile beinhalten die Berechtigungen.
Daher ist die aktuelle Berechtigungsstruktur hierarchisch bei der Helvetia.


\section{IAM}
\label{subsec:IAM}
Die Virginia IT Agency beschreibt IAM wie folgt, dass \ac{IAM} die Möglichkeit ist, sämtliche Nutzer und Profile, welche man zu den jeweiligen Personen über die IT-Umgebung über Nutzerrollen und Businessregeln zu ordnen kann, zu handhaben. Dabei ist die Zugriffverwaltung die Möglichkeit die Zugriffskontrolle Regeln über verschiedene Plattformen einzuhalten. Ein wichtiger Teil von \ac{IAM} dabei ist sicherzustellen, dass die Nutzer einen sicheren Zugriff auf die Ressourcen haben und auch nur die Ressourcen, die sie benötigen, um ihre Arbeit zu erledigen. \cite{Virg07}
\begin{figure}[h!]
 \centering
 \includegraphics[width=1\textwidth]{gfx/Picture/IAM.PNG}
 \caption{Übersicht von IAM \cite{Moha19}}
 \label{fig:IAM}
\end{figure}
Im Bild (\ref{fig:IAM}) kann man erkennen, dass wenn ein Nutzer oder Programmzugriff auf eine Ressource möchte, dass dieser die Identifizierung, Authentifizierung, Autorisierung  sowie Rechtschaffenheit vorlegen und einhalten muss. \cite{Moha19}
\newline
Im Rahmen dieser Ausarbeitung handelt es sich bei den Ressourcen, um die Berechtigungen (\ref{sec:Berechtigung}).
