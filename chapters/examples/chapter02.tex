\chapter{Recherche}
\label{ch:Recherche}
In diesem Kapitel geht es um mehrere selbstdurchgeführte Befragungen zur Berechtigungsstruktur und der jährlichen Rezertifizierung.
Die jährliche Rezertifizierung ist die Überprüfung, ob die Mitarbeiter ihre Berechtigungen benötigen oder nicht.
Diese wird von den Teamleitern und Führungskräften durch geführt, welche die Berechtigungstruktur nutzen, um dies zu tun.
Dafür wurden drei verschiedene Gruppen befragt.
Die Ergebnisse sind die Grundlage für die Notwendigkeit eines sicheren und übersichlichen Struktur sowie die Festellung der Hauptprobleme der bestehenden Struktur.

\section{Vorgehensweise}
\label{sec:Vorgehensweise}
Für die Befragung wurde erstmal analysiert welche Stakeholder es gibt.
Bei der Analyse wurden die folgenden drei Stakeholder festgestellt:

\begin{itemize}
	\item \ac{FuT}
	\item Entwickler
	\item Mitarbeiter
\end{itemize}
Die \ac{FuT} entsprechen den Stakeholdern, die die Berechtigungsstruktur für die jährliche Rezertifizierung nutzen.
Diese haben für eine hohe Priorität, da diese die Berechtigungsstruktur direkt verwenden und für die Sicherheit der Struktur gewähren.
\newline
Die Entwickler sind die Stakeholder, die an der Berechtigungsstruktur gearbeitet haben.
Ebenso wie die Teamleiter und Führungskräfte haben die Entwickler auch eine hohe Priorität, weil diese die Struktur warten und verändern.
\newline
Die Mitarbeiter umfassen das restlichen Arbeitspersonal.
Im Gegensatz zu den anderen beiden Stakeholder haben die Mitarbeiter eine geringe Priorität, da diese weder die Struktur nutzen noch einen anderen Kontakt haben.
\newline
\newline
Nachdem diese drei Stakeholder festgestellt worden sind, wurde spezifisch für diese drei Fragenkataloge entwickelt.
Dabei ist das Ziel festzustellen, welche die größten Probleme aus der Sicht der Teamleiter und Führungskräfte sowie Entwickler gibt.
Die Mitarbeiter wurden befragt, wie viele Berechtigungen diese verfügen, die sie eigentlich nicht mehr brauchen.
Die große Herausforderung besteht dabei, die richtigen Fragen für die Fragenkataloge zu entwerfen.
Die Institution \ac{PRC} hat festgestellt, dass bei geschlossenen Fragen die befragt zu einem großen Teil (über 90\%) eine der vorgeschlagenen Antwortet gewählt haben. \cite{Survey}
Dies stellt ein Problem dar.
Wenn die Fragen zu geschlossen sind, dann kann es dazu führen, dass die befragten nicht die Probleme angeben, die sie sehen.
Auf der anderen Seite sind zu offene Fragen auch eine Herausforderung, da es schwierig wird die verschiedenen Antworten zu quantifizieren und auswerten zu können.
Ebenso ist die Wortwahl ein entscheidener Faktor.
In einer Studie von \ac{PRC} in 2003 wurden die Personen befragt, ob diese für oder den Krieg in Iraq sind, um Saddam Hussein's herrschaft zu beenden.
68\% haben ja gesagt und 25\% für nein.
Darauf wurde die Frage geändert zu, ob diese für oder den Krieg in Iraq sind, um Saddam Hussein's herrschaft zu beenden, selbst wenn es tausende Verluste gibt.
Mit dieser Änderung haben nur noch 43\% dafür gestimmt und 48\% dagegen. \cite{Survey}
\newline
Dies ist relevant, da zum Beispiel bei der Befragung der Mitarbeiter bei einer falschen Formulierung den Gedanken bekommen könnten, dass diese mit der Beantwortung der Frage ihr in Ordnung geben, dass ich ihnen die genannten Berechtigungen entfernen.
Dies kann zu fehlerhaften Ergebnissen führen.
Deshalb müssen die Fragen gut überlegt sein.
\newline
\newline
Für die \ac{FuT} wurden die folgenden Fragen überlegt: 
\begin{itemize}
	\item Wie handhabbar ist für Sie der aktuelle Prozess für die jährliche Rezertifizierung der Vorgangsberechtigung Ihrer Mitarbeiter?
	\item Was finden Sie im aktuellen Rezertifizierungsprozess gut?
	\item Was finden Sie im aktuellen Rezertifizierungsprozess schlecht?
	\item Was würden Sie gerne am aktuellen Rezertifizierungsprozess ändern?
	\item Was halten Sie von der Idee das Profile entweder nur noch (Unter-)Profile oder Profile ausschliesslich Berechtigungen beinhalten?
	\item Soll es eine einheitliche Strukturierung für die Berechtigungsstruktur innheralb der Bereiche geben?
	\item Haben Sie weitere Anmerkungen?
\end{itemize}
Die Fragen wurden größtenteils offen Formuliert, um am besten die Probleme am bestehenden System zu finden.
Dabei umfassen die ersten drei Fragen den Ist-Zustand der Berechtigungstruktur und wie dies die \ac{FuT} finden.
Fragen vier, fünf und sechs wie der Soll-Zustand der Berechtigungstruktur werden soll.
Dabei wurden auch konkrete Vorschläge in den Fragen unterbreitet, um einen erst Eindruck von den \ac{FuT} zu erhalten.
Zum Schluss gab es noch die offene Frage, ob es weitere Anmerkungen gibt, um eventuelle Antworten und Anmerkung zu erhalten, die durch die vorherigen Fragen nicht abgedeckt wurden.
\newline
\newline
Für die Entwickler wurden die folgenden Fragen überlegt: 
\begin{itemize}
	\item Welche Erfahrung haben Sie mit der Berechtigungsstruktur gehabt?
	\item Auf welche Probleme sind Sie im Zusammenhang mit der Berechtigungsstruktur gestoßen?
	\item Gibt es Sachen, die man bei der Berechtigungsstruktur beachten muss?
	\item Haben Sie Vorschläge wie man die Berechtigungsstruktur besser gestalten könnte?
	\item Haben Sie weitere Anmerkungen?
\end{itemize}
Die erste Frage soll dazu Anregen sich über das Thema Gedanken zu bilden, damit die folgenden Fragen einfacher zu beantworten sind.
Dabei soll die zweite Frage klarstellen welche Herausforderungen die befragte Person mit der Struktur hatte.
Dies ermöglicht es dann präventiv gegen diese Vorzugehen und dies direkt mit in die neuen Konzepte zu integrieren.
Bei der dritten Frage sollen mögliche ausnahme Fälle genannt werden, die bei einem Konzept mit berücksichtigt werden müssten.
Anschließend wird gefragt, ob die Person eventuel sich selber Gedanken gemacht hat, welche Möglichkeiten es gibt, um die aktuelle Struktur zu verbessern.
Zum Abschluss wird wieder gefragt, ob es weitere Anmerkung gibt, um falls es solche gibt aber nicht von den vorherigen Fragen abgedeckt wurde.
\newline
\newline
Die Mitarbeiter haben die Frage bekommen:
\newline
\newline
\textit{Wie viele Vorgangsberechtigungen in der Produktion Sie im Hoblink haben, die Sie nicht mehr nutzen?}
\newline
\newline
Bei dieser Frage ware es wichtig diese so zu formulieren, dass die befragte Person nicht den Eindruck bekommt, dass ich ihr die Berechtigungen wegnehmen möchte.
Dies würde ansonsten zu fehlerhaften Ergebnissen führen.
Die Befragung der Mitarbeiter dient dazu, um die Problematik der nicht optimalen Berechtigungsstruktur darzustellen und auch um feste Zahlen zu haben.

\section{Auswertung}
\label{sec:Auswertung}


\section{Ergebnis}
\label{sec:Ergebnis}