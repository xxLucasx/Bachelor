\chapter{Recherche}
\label{ch:Recherche}
In diesem Kapitel geht es um mehrere selbstdurchgeführte Befragungen zur Berechtigungsstruktur und der jährlichen Rezertifizierung.
Die jährliche Rezertifizierung ist die Überprüfung, ob die Mitarbeiter ihre Berechtigungen benötigen oder nicht.
Diese wird von den Teamleitern und Führungskräften durch geführt, welche die Berechtigungstruktur nutzen, um dies zu tun.
Dafür wurden drei verschiedene Gruppen befragt.
Die Ergebnisse sind die Grundlage für die Notwendigkeit eines sicheren und übersichlichen Struktur sowie die Festellung der Hauptprobleme der bestehenden Struktur.

\section{Vorgehensweise}
\label{sec:Vorgehensweise}
Für die Befragung wurde erstmal analysiert welche Stakeholder es gibt.
Bei der Analyse wurden die folgenden drei Stakeholder festgestellt:

\begin{itemize}
	\item Teamleiter und Führungskräfte
	\item Entwickler
	\item Mitarbeiter
\end{itemize}

Die Teamleiter und Führungskräfte entsprechen den Stakeholdern, die die Berechtigungsstruktur für die jährliche Rezertifizierung nutzen.
Diese haben für eine hohe Priorität, da diese die Berechtigungsstruktur direkt verwenden und für die Sicherheit der Struktur gewähren.
\newline
Die Entwickler sind die Stakeholder, die an der Berechtigungsstruktur gearbeitet haben.
Ebenso wie die Teamleiter und Führungskräfte haben die Entwickler auch eine hohe Priorität, weil diese die Struktur warten und verändern.
\newline
Die Mitarbeiter umfassen das restlichen Arbeitspersonal.
Im Gegensatz zu den anderen beiden Stakeholder haben die Mitarbeiter eine geringe Priorität, da diese weder die Struktur nutzen noch einen anderen Kontakt haben.
\newline
\newline
Nachdem diese drei Stakeholder festgestellt worden sind, wurde spezifisch für diese drei Fragenkataloge entwickelt.

\section{Auswertung}
\label{sec:Auswertung}


\section{Ergebnis}
\label{sec:Ergebnis}