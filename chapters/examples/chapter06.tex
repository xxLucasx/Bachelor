\chapter{Fazit}
\label{ch:chapter06}
In der Durchführung der Arbeit wurden die verschiedenen Stakeholder befragt, um herauszufinden, was das Problem ist.
Dabei wurde es klar, dass die Ursachen für das Problem vielfältig sind, wie man im Ursachen-Wirkungs-Diagramm (\ref{fig:Fisch}) erkennen kann.
Diese verschiedenen Ursachen benötigen individuelle Lösungen.
Daher wurde in dieser Arbeit nur die Ursachen im Punkt Methode begutachtet.
Um diese zu beheben, wurden die Kritiken von den IT-Spezialisten und \ac{FuT} sowie Best-Practises von Unternehmen betrachtet, um daraus eine ideale Lösung zu finden.
Dabei wurde klar, dass das Befragen von IT-Spezialisten und \ac{FuT} aufwendig ist für eine einzelne Person und mehr Zeit eingeplant werden sollte für diesen Schritt.
Daraus wurden die beiden Konzepte hierarchische Struktur und minimalistische Struktur entwickelt.
Mithilfe der Prioritätsanalyse und der Nutzwertanalyse wurde diese beiden bewertet.

\section{Empfehlung der Konzeptwahl}
\label{sec:chapter06}
Die beiden Konzepte hierarchischen Struktur sowie minimalistische Struktur haben ihre Vor- und Nachteile, die je nach Umgebung besser oder schlechter geeignet sind.
Im Falle der Helvetia ist zum Beispiel die hierarchische Struktur besser geeignet.
\newline
Die Implementierung der hierarchischen Struktur ist aufwendiger als die des minimalistischen Konzepts, aber dafür ist diese übersichtlicher.
Das minimalistische Konzept eignet sich mehr für eine Berechtigungsstruktur, die über weniger Berechtigungen verfügt, da ansonsten die Anzahl der individuellen Berechtigungen, die ein Nutzer hat, zu unübersichtlich wird.
Bei einem kleinen Unternehmen, welches zum Beispiel nur über 50 Berechtigungen verfügt, wäre das minimalistische Konzept besser geeignet als das Konzept der hierarchischen Struktur, da bei einem solch kleinen Unternehmen die Mitarbeiter an verschiedenste Aufgaben arbeiten müssen.
Daher sollte das Berechtigungskonzept einfach genug gestaltet sein, sodass die Vergabe und das Entfernen von individuellen Berechtigungen so leicht wie möglich ist.
Auf der anderen Seite braucht ein größeres Unternehmen wie die Helvetia mehr standardisierte Profile, da es nicht die Zeit und Ressourcen gibt, für jeden einzelnen Mitarbeiter ein individuelles Profil zu erstellen.
Zudem muss auch berücksichtigt werden, dass diese Profile und Berechtigungen vermerkt werden müssen.
Dies könnte sich als eine Herausforderung herausstellen, wenn ein Mitarbeiter über 100 zusätzliche Berechtigungen verfügt.
\newline
Außerdem ist die hierarchische Struktur im Bereich der Performance besser als das minimalistische Konzept.
Dies ist allerdings kein Aspekt gewesen, welcher beim Projektantrag erwähnt wurde.
Dennoch ist es von einer hohen Wichtigkeit, dass eine Versicherung in der Lage ist, zu jeder Zeit auf ihre Verträge sowie andere wichtige Dokument zugreifen zu können.
\newline
Zudem sollte auch die Wartung einfacher verlaufen, da es schneller auffällt, wenn ein Profil an einer Stelle hinzugefügt wurde, an der es nicht sein sollte.
Dadurch wird die Wahrscheinlichkeit verringert, dass die Berechtigungsstruktur in der Zukunft unkontrolliert weiter wächst.
Deswegen empfehle ich das Konzept der hierarchischen Struktur, da diese den Anforderungen besser genügt.
Ebenso schätze ich die Langlebigkeit des Konzeptes höher ein sowie entspricht die hierarchische Struktur auch eher dem Best Practice von Datenbanken.

\section{Zusammenfassung und Ausblick}
\label{sec:chapter06:Ausblick}
Wie sich während der Arbeit herausgestellt hat, ist das Entwickeln von Konzepten für eine Berechtigungsstruktur keine einfache Aufgabe.
Um ein geeignetes Konzept zu entwickeln, muss erstmal festgestellt werden, was die aktuellen Probleme sind.
Methoden wie Befragungen und Umfragen kosten viel Zeit und sind aufwendig zu gestalten, um so akkurate Ergebnisse wie möglich zu erhalten.
Außerdem muss sich informiert werden, welche Grundvoraussetzungen erfüllt werden müssen, die entweder vom Unternehmen oder der gesetzlichen Seite gestellt werden.
Dabei variiert auch die Evaluierung der Konzepte, da sich dies he nach Anforderung ändern kann.
\newline
Bei der Arbeit wurden hauptsächlich Definitionen sowie Best Practises verwendet, da es wenig Material zum Thema Entwicklung von Konzepten für Berechtigungsstrukturen gibt, wobei das Best Practise immer möglich ist umzusetzen.
Dabei ergibt sich die Fragen, ob infolge dieser Arbeit die genannte Vorgehensweise auch für andere Strukturen genutzt werden kann.
\newline
Zudem muss in der Zukunft weiterhin auf die Berechtigungsstrukturen geachtet werden.
Denn diese haben eine wichtige Rolle im Bereich der Versicherungen wie den Banken.
Sollte es dabei zu Problemen kommen, hätte dies einen hohen finanziellen Preis.
In der Zukunft ist es möglich, dass es Tools geben wird, die automatisch solche Strukturen generieren, basierend an den Anforderungen.
Natürlich müssten diese generierten Strukturen von Personen auf Fehler überprüft werden. Aber ein Tool könnte bessere Strukturen für individuelle Problem generieren im Vergleich zu einem Menschen.