\chapter{Fazit}
\label{ch:chapter06}

\section{Empfehlung}
\label{sec:chapter06}
Die beiden Konzepte hierarchische Struktur sowie Minimalistisch haben ihr Vor und Nachteile.
Die je nach Umgebung besser und schlechter geeignet sind.
Im Falle der Helvetia ist zum Beispiel die hierarchische Struktur besser.
\newline
Die Implementierung ist definitiv aufwendiger als die vom Minimalistisch Konzept, aber dafür ist diese übersichtlicher.
Das Minimalistisch Konzept eignet sich mehr für eine Berechtigungsstruktur, die über weniger Berechtigungen verfügt, da ansonsten die Anzahl der individuellen Berechtigungen ein Nutzer hat zu unübersichtlich wird.
Bei einem kleinen Unternehmen, welches zum Beispiel nur über 50 Berechtigungen verfügt, wäre das Minimalistisch Konzept besser geeignet als das hierarchische Struktur Konzept, da bei einem solchen kleinen Unternehmen die Mitarbeiter für verschiedenste Aufgaben arbeiten müssen.
Daher sollte das Berechtigungskonzept einfach genug gestaltet sein, dass die Vergabe und entfernen von individuellen Berechtigungen so leicht wie möglich ist.
Auf der anderen Seite braucht ein größeres Unternehmen wie die Helvetia mehr standardisierte Profile, da es nicht die Zeit und Ressourcen gibt für jeden einzelnen Mitarbeiter ein individuelles Profil zu erstellen.
Zudem muss auch berücksichtigt werden, dass diese Profile und Berechtigungen vermerkt werden müssen.
Dies könnte sich als eine Herausforderung stellen, wenn ein Mitarbeiter über 100 zusätzliche Berechtigungen verfügt.
\newline
Außerdem ist die hierarchische Struktur im Bereich der Performance besser also als das Minimalistisch Konzept.
Dies ist zwar kein Aspekt gewesen, auf welches bei dem Projektantrag erwähnt wurde.
Dennoch ist es von einer hohen Wichtigkeit, dass eine Versicherung in der Lage ist zur jeden Zeit auf ihre Verträge sowie andere wichtige Dokument zugreifen zu können.
\newline
Zudem sollte auch die Wartung dieser einfacher verlaufen, da es schnell auffällig wird, wenn ein Profil hinzugefügt wurde an einer Stelle, welches nicht da sein sollte.
Dadurch wird die Wahrscheinlichkeit verringert, dass die Berechtigungsstruktur in der Zukunft so weit wächst.
Deswegen empfehle ich die hierarchische Struktur Konzept über die Minimalistisch Konzept, da diese den Anforderung besser genüge wird sowie der Langlebigkeit des Konzeptes.
\newpage

\section{Ausblick}
\label{sec:chapter03:grafiken:minipage}
Donec gravida consequat arcu, et mollis tortor posuere vitae. Sed pharetra turpis a ante commodo accumsan. Suspendisse leo nulla, accumsan sit amet dapibus in, posuere eget turpis. Vivamus enim sapien, porta id placerat eget, laoreet sed massa. Class aptent taciti sociosqu ad litora torquent per conubia nostra, per inceptos himenaeos.
