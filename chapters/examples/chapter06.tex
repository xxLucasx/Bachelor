\chapter{Fazit}
\label{ch:chapter06}

\section{Empfehlung}
\label{sec:chapter06}
Die beiden Konzepte hierarchische Struktur sowie Minimalistisch haben ihr Vor- und Nachteile.
Die je nach Umgebung besser oder schlechter geeignet sind.
Im Falle der Helvetia ist zum Beispiel die hierarchische Struktur besser geeignet.
\newline
Die Implementierung ist definitiv aufwendiger als die des minimalistischen Konzepts, aber dafür ist diese übersichtlicher.
Das minimalistische Konzept eignet sich mehr für eine Berechtigungsstruktur, die über weniger Berechtigungen verfügt, da ansonsten die Anzahl der individuellen Berechtigungen, die ein Nutzer hat, zu unübersichtlich wird.
Bei einem kleinen Unternehmen, welches zum Beispiel nur über 50 Berechtigungen verfügt, wäre das minimalistische Konzept besser geeignet als das Konzept der hierarchischen Struktur, da bei einem solch kleinen Unternehmen die Mitarbeiter für verschiedenste Aufgaben arbeiten müssen.
Daher sollte das Berechtigungskonzept einfach genug gestaltet sein, dass die Vergabe und das Entfernen von individuellen Berechtigungen so leicht wie möglich ist.
Auf der anderen Seite braucht ein größeres Unternehmen wie die Helvetia mehr standardisierte Profile, da es nicht die Zeit und Ressourcen gibt, für jeden einzelnen Mitarbeiter ein individuelles Profil zu erstellen.
Zudem muss auch berücksichtigt werden, dass diese Profile und Berechtigungen vermerkt werden müssen.
Dies könnte sich als eine Herausforderung herausstellen, wenn ein Mitarbeiter über 100 zusätzliche Berechtigungen verfügt.
\newline
Außerdem ist die hierarchische Struktur im Bereich der Performance besser als das minimalistische Konzept.
Dies ist allerdings kein Aspekt gewesen, welcher beim Projektantrag erwähnt wurde.
Dennoch ist es von einer hohen Wichtigkeit, dass eine Versicherung in der Lage ist, zu jeder Zeit auf ihre Verträge sowie andere wichtige Dokument zugreifen zu können.
\newline
Zudem sollte auch die Wartung einfacher verlaufen, da es schneller auffällt, wenn ein Profil an einer Stelle hinzugefügt wurde, an der es nicht sein sollte.
Dadurch wird die Wahrscheinlichkeit verringert, dass die Berechtigungsstruktur in der Zukunft unkontrolliert weiter wächst.
Deswegen empfehle ich das Konzept der hierarchischen Struktur, da diese den Anforderungen besser genügt. Ebenso schätze ich die Langlebigkeit des Konzeptes höher ein.
\newpage

\section{Ausblick}
\label{sec:chapter03:grafiken:minipage}
Donec gravida consequat arcu, et mollis tortor posuere vitae. Sed pharetra turpis a ante commodo accumsan. Suspendisse leo nulla, accumsan sit amet dapibus in, posuere eget turpis. Vivamus enim sapien, porta id placerat eget, laoreet sed massa. Class aptent taciti sociosqu ad litora torquent per conubia nostra, per inceptos himenaeos.
