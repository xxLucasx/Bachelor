%*******************************************************
% Abstract in German
%*******************************************************
\begin{otherlanguage}{ngerman}
	\pdfbookmark[0]{Zusammenfassung}{Zusammenfassung}
	\chapter*{Zusammenfassung}
In der heutigen Welt wird für alles eine Berichtigung verlangt.
Sei es das Anmelden vom Computer oder Handy bis zu Chipkarten, um in gewisse Gebäude reinzukommen.
Dabei haben Mitarbeiter in Unternehmen spezifische Berechtigungen, um ihre Arbeit zu erfüllen.
Die Anzahl dieser Berechtigungen sind im Dreistelligen.
Diese müssen verwaltet werden, da es problematisch ist, wenn ein Mitarbeiter mehr Berechtigungen hat als, die er benötigt.
Daher braucht das Unternehmen eine Berechtigungsstruktur, die sowohl sicher ist, als auch den Mitarbeitern alle benötigten Berechtigungen zur Verfügung stellt.
\newline
Am Anfang wird sich der Ist-Zustand im Unternehmen betrachtet.
Dabei werden verschiedene interessen Gruppen befragt, um die Probleme der aktuellen Struktur festzustellen.
Diese belaufen sich auf \ac{FuT}, IT-Spezialisten und die anderen Mitarbeiter, um verschiedene Blickwinkel und die Größe für das Problem zu bekommen.
Anschließend wird der Stand der Technik betrachtet.
Es gibt wenige Arbeiten für diesen Bereich.
Daher wurden die Best-Practise von verschiedenen Unternehmen wie Microsoft oder IBM betrachtet.
Zudem wird auch eine parallele zu Datenbanken gezogen, da diese auch über Berechtigungsstrukturen verfügen.
Die daraus erstellten Konzepte werden mittels Prioritätsanalyse und Nutzwertanalyse ausgewertet, um daraus zu bestimmen, welches besser in diesem Fall für die Helvetia geeignet ist.
\newline
Dabei hat sich herausgestellt, dass nicht alle Best-Practise umgesetzt werden können, da diese nicht für die Anforderung für das Unternehmen ausreichen.
Zum Beispiel sollen Nutzer idealerweise nicht einzel Berechtigungen erhalten.
Dies ist jedoch nicht umsetztbar, da entweder für jeden einzelnen Fall ein neues Standardprofil erstellt werden müsste oder der Mitarbeiter mehr Berechtigungen erhält, als die er benötigt.
Daher müssen Kompromisse gemacht werden, um die ideale Berechtigungsstruktur für das Unternehmen zu entwickeln.
\end{otherlanguage}
